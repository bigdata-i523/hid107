\documentclass[sigconf]{acmart}

\usepackage{hyperref}

\usepackage{endfloat}
\renewcommand{\efloatseparator}{\mbox{}} % no new page between figures

\usepackage{booktabs} % For formal tables

\settopmatter{printacmref=false} % Removes citation information below abstract
\renewcommand\footnotetextcopyrightpermission[1]{} % removes footnote with conference information in first column
\pagestyle{plain} % removes running headers

\begin{document}
\title{DevOps In Support Of Big Data Applications And Analytics}


\author{Juan Ni}
\affiliation{%
  \city{Bloomington} 
  \state{Indiana} 
  \postcode{47401}
}
\email{nijuan@iu.edu}




\begin{abstract}
We show the relationship between devops and big data applications and analytic. The investigation will focus how devops support big data application. We Also find out the how devops on cloud computing platform impact big data application, since the carrier for  most of big data application and analytic is cloud platform. At last, defining the difficulty of implement devops which is important for designing a big data application.
\end{abstract}

\keywords{i523, HID 107, big data, devops, cloud platform}


\maketitle

\section{Introduction}

Following with cloud computing and big data analytic prosperous, devops begain getting more and more attention. In the traditional development group, we usually have five processes which are ``Analyze'', ``Design'' , ``Code'', ``Test'', and ``Maintain''\cite{traditional:01}. According to Justin's idea, each of the traditional process is isolate with others, once one of the process group finish their part, they will simply shift the work to next group for the next level processing. I have been participated into a traditional development, once the client change their needs, traditional will take really long time to track back the work and making adjustment at traditional development team. Especially for big data application, due to the complex of big data application, the development period will be longer than usual, therefore, we need devops to help us satisfy our client needs at long period development and keep efficiency processing speed. Devops is not a set of tool, it is a methodology that include basic principle and practical. According to the concept of devops, the processes of devops are include ``Code'', ``Build'', ``est'', ``Package'', "Release'', "Configure", and "Monitor" \cite{wiki:01}. Unlike the traditional development processes, all processes of devops are connecting inside a loop, this make devops as an integration of Development and Operations.
 
\section{The need of devops for big data application}

The gap of communication between development and operations on big data application is the main issue. Andrew points out that ``The idea of DevOps is to tear down the silos between software developers and IT infrastructure administrators to make sure everyone is focused on a singular goal.'' \cite{bigdata:01} In the traditional development team, developers are not involve into the analyst activity, because the big data analyst and application developer is isolated. According to my experience, once the developer get the changing requirement from analytic team, they need to take time to understand the adjustment and reorganized the manpower inside the group. This communication delay will lower the entire processing speed, decreasing the competitiveness of big data analysis. Andre mention that ``IT leaders are under increased pressure to produce results. This forces analytics scientists to revamp their algorithms. These major changes in analytic models often require drastically different infrastructure resource requirements than was originally planned for'' \cite{bigdata:01} In big data application and analytic project, analyst change their algorithm ceaselessly, and the change of analytically model will make the infrastructure and resource demeaned become much different with the original one.  We know that data is timeliness, big data is not a exception. If big data application processing take to much time, the outcome will be less value, so big data application need devops to prevent data losing value by fall behind. 

\section{The values of devops}

The main value of devops is to break down the ``Wall of confusion'' between developer and operator. According to Jerome's idea, devops have two main values which are “Continuous Delivery” and “Benefits” \cite{devops:01} . The ``Benefits'' of devops include but not limit to ``Repeatability and Reliability'', ``Productivity'', ``Time to recovery'', "Guarantee that infrastructure is homogeneous", "Make sure standards are respected", and ``Allow developer to do lots of tasks themselves''. Those benefit allow developer and operator working better as a team, and understnad each others work; according to Allerin's idea, . ``Continuous Delivery'' help project team decrease the application delivery period by having faster application development, high frequency update can reduce the risk and cost of changing demand during the delivery. This is extremely useful for big data application because it always requirement lot of change during delivery. The increasing of delivery frequency can let the project team more familiar with the processes of application deployment, also will getting more feedback from the user. Therefor, Haff points out the core value of devops which is ``When DevOps began, so did a shorthand description for the model: It broke down the wall between dev and ops. The teams communicated better and operated with a shared set of objectives and concerns. At the extreme, there were no longer devs and ops people, but DevOps skill sets.'' \cite{devops:05}.

IBM organizes the values of devops into three domains, ``increase customer experience'', ``improve innovation ability'', and ``faster achieve value''. \cite{IBM:01}. According to IBM's concept, Devops is not the goal of application development, but it can let development team reach their goal. It increase customer experience via faster update, and having faster response to customer's feedback. Then we using devops to avoid rework cause by misunderstanding the demand, so the project team have time and  energy to investigate new technology. Finally, once the delivery period is shorter and shorter, user can actually use the application early before the content inside the application are out date, this is important for big data application because the replacement of big data analysis algorithm is changing all the time. The above values show us that Devops is endless, it will continually develop the entire project team's technology, processes flow, teamwork, and team culture. 



\section{Cloud platform and devops}
Cloud platform play a really important role in big data application designing. According to Microsoft concept, ``Cloud Platform System lowers costs at all stages of the infrastructure life-cycle''\cite{bigdata:03};The expectation of cloud platform is change the capital cost to operate cost, company don't need to figure out the cost of hardware for building a cloud server for the big data application, they can use public cloud platform really economical to prevent any waste on the computing ability. According to Allerin's idea, ``enterprises are now considering of moving their Big data and Hadoop projects to public cloud services for gaining the much-needed agility they need for their data scientists.With a scalable and flexible infrastructure platform, IT organizations and development team together can spin up virtual Hadoop or Spark clusters within minutes.'' \cite{bigdata:02}, so the true value of using cloud platform is decrease the barrier which slow down the development speed and developer time. Sumologic's article describe how how automatic devops works in the some mainstream cloud platforms, according to his idea,  `` The delivery pipeline collapses to a single sile where developers, testers, and operations professionals collaborate as one and as much of the deployment process as possible is automated.''\cite{bigdata:04}; most of devops work on the cloud platform can be automatic, so development team can be free from the heavy lifting of daily work such as management hardware and patch installing. Therefore, using cloud platform mean company from providing product change to providing service, which fit the big data application purpose. I think Customer won't buy the application if the content doesn't make sense, so let cloud platform supporting devops  is good for the project team focus on the content inside the application, and be able to development more efficiently. 

Mary mention that ``In a true Waterfall development project, each of these represents a distinct stage of software development, and each stage generally finishes before the next one can begin.''\cite{agility},that mean user expect having high quality function and continuous update in the same time. Unlike the traditional devilment team provide ``Waterfall methodology'' to release event, devops provide frequent release events can satisfy customer need and maximize utilize the automation of cloud platform. Moreover, According to Nelson and Raouf's idea,``Big data is the term for a collection of data sets so large and complex that it becomes difficult to process using on-hand database management tools or traditional data processing applications ''\cite{platform}, this show the great compatibility between cloud computing platform and big data application. We know that devops are more often use for could platform. So, following the supporting by cloud platform, Devops make operate and development keep almost the same speed and flexibility which is suitable for big data business need in cloud platform.

\section{The weakness of devops}

Haff mention that``Dev teams that are making the most of this model need to focus on improved application architectures and developer workflows'' \cite{devops:05}, this is seems pretty easy but actually it has difficulty of implement, and it is the biggest weakness of devops. According to my working experience, each company have their own development tool and process, they all have their own feature, so it is really hard to have a temple that fit to every company and situation for devops. Futhermore, Jeff provided a really nice definition for the developer in devops group which is ``"DevOps" is meant to denote a close collaboration and cross-pollination between what were previously purely development roles, purely operations roles, and purely QA roles. Because software needs to be released at an ever-increasing rate, the old "waterfall" develop-test-release cycle is seen as broken. Developers must also take responsibility for the quality of the testing and release environments.'' \cite{kill}, this mean if a traditional developer want to fit into a devops team, that developer need to fimilar with testing、application implement、and need which almost cover every role's job inside the team. I think a person's energy and time is limited, if a person spend time to something, then that person spend time to other thing will be decreasing. This is same for developer, if a developer have ``Many hats ''\cite{kill}, then that developer  probably no be able to focus on coding. And this will cause ``Jack of All Trades, Master of None" \cite{kill} happen to developer, thus even the product release spedding is  increasing but the quality is low. According to my experience on a big application team, for team leader, they only able to organized team and develop processes under limit source. So it is hard to having a optimal plan at most of time, for example, to let each member inside the team have same understanding and sense of duty is really hard, but the above section mention that breaking the wall between develop and operate require team member familiar with each other's area, so the difficulty of devops is to fill the knowledge gap between each team member inside the group. 

\section{Conclusions}

The above section show the main purpose of doing devops in big data applications and analytic is to eliminate the isolate situation between Solution Architect(big data analyst) and programmer.This can be achieve by cross education training for both architect and programmer, so they can understand the basic concept and terminology from both domain. Once they finish the training, they will suppose to have a better understanding of each others' idea, and prevent the process off track. Furthermore, devops can let those two group testing the application environment and adjusting the foundation framework to meet the new needs, it represent faster fixing and update capacity. 

\section{Acknowledgement}

I would like to  take this chance to thanks to my tutor Juliette Zerick, in process on reviewing my paper, she gave me many useful comment and advise. At the same time, I would like to thanks my instructor laszewsk, give me useful knowledge about how to write a report on Latex format. Finally, I would love to thanks my internship supervisor who give me many idea about my topic.






\bibliographystyle{ACM-Reference-Format}
\bibliography{report} 

\end{document}
